%	------------------------------------------------------------------------------
%
%		작성  : 2020년 
%				11월 
%				06일 
%				첫 작업
%
%

%	\documentclass[25pt, a1paper]{tikzposter}
%	\documentclass[25pt, a0paper, landscape]{tikzposter}
%	\documentclass[20pt, a1paper ]{tikzposter}
	\documentclass[	20pt, 
							a1paper, 
%							portrait, %
							landscape, %
							margin=0mm, %
							innermargin=10mm,  		%
							blockverticalspace=4mm, %
							colspace=5mm, 
							subcolspace=0mm
							]{tikzposter}




%	\documentclass[25pt, a1paper]{tikzposter}
%	\documentclass[25pt, a1paper]{tikzposter}
%	\documentclass[25pt, a1paper]{tikzposter}

% 	12pt  14pt 17pt  20pt  25pt
%
%	a0 a1 a2
%
%	landscape  portrait
%

	%% Tikzposter is highly customizable: please see
	%% https://bitbucket.org/surmann/tikzposter/downloads/styleguide.pdf

	%	========================================================== 	Package
		\usepackage{kotex}						% 한글 사용


%% Available themes: see also
%% https://bitbucket.org/surmann/tikzposter/downloads/themes.pdf
%	\usetheme{Default} 		% 블럭 제목이 안나옴
%	\usetheme{Rays}		% 블럭 제목이 센터 정렬
%	\usetheme{Basic}		% 블럭 제목이 안나옴
	\usetheme{Simple}		% 통상 사용
%	\usetheme{Envelope}	% 블럭 제목이 안나옴
%	\usetheme{Wave}		% 전체 제목이 너무 공간을 많이 차지
%	\usetheme{Board}		% 블럭 제목이 센터 정렬
%	\usetheme{Autumn}		% 깔끔 블럭 제목이 흐림
%	\usetheme{Desert}		% 블럭 제목이 흰색	

%% Further changes to the title etc is possible
%	\usetitlestyle{Default}			%
%	\usetitlestyle{Basic}				%
%	\usetitlestyle{Empty}				%
%	\usetitlestyle{Filled}				%
%	\usetitlestyle{Envelope}			%
%	\usetitlestyle{Wave}				%
%	\usetitlestyle{verticalShading}	%


%	\usebackgroundstyle{Default}
%	\usebackgroundstyle{Rays}
%	\usebackgroundstyle{VerticalGradation}
%	\usebackgroundstyle{BottomVerticalGradation}
%	\usebackgroundstyle{Empty}

%	\useblockstyle{Default}
%	\useblockstyle{Basic}
%	\useblockstyle{Minimal}		% 이것은 간단함
%	\useblockstyle{Envelope}		% 
%	\useblockstyle{Corner}		% 사각형
%	\useblockstyle{Slide}			%	띠모양  
	\useblockstyle{TornOut}		% 손그림모양


	\usenotestyle{Default}
%	\usenotestyle{Corner}
%	\usenotestyle{VerticalShading}
%	\usenotestyle{Sticky}

%	\usepackage{fontspec}
%	\setmainfont{FreeSerif}
%	\setsansfont{FreeSans}

%	------------------------------------------------------------------------------ 제목

	\title{김대희 식당}

	\author{ 2020년 
				11월
				07일  }

%	\institute{서영엔지니어링}
%	\titlegraphic{\includegraphics[width=7cm]{IMG_1934}}

	%% Optional title graphic
	%\titlegraphic{\includegraphics[width=7cm]{IMG_1934}}
	%% Uncomment to switch off tikzposter footer
	% \tikzposterlatexaffectionproofoff

%[중앙] 
%[수정] 
%[구덕]
%[남구]
%[부전]
%[시민]



\begin{document}

	\maketitle

	\begin{columns}

	%	====== ====== ====== ====== ====== 
		\column{0.2}

%	------------------------------------------------------------------------------ 국수
%			\block[titleleft,linewidth=3mm]
			\block[
%						titleoffsetx =0mm,	 	%
%						titleoffsety=0mm,	 	% 
%						bodyoffsetx=00mm,	%
						bodyoffsety=20mm	 	%
%						titlewidthscale=2, 
%						bodywidthscale=1,
						titleleft, 
%						titlecenter, 
%						titleright,
%						bodyverticalshift=0mm  % 제목과 본과의 간격
%						roundedcorners=50, 
						linewidth=3mm,
%						titleinnersep=10mm, 
%						bodyinnersep=0mm
					]
			{■ 국수 }



%	------------------------------------------------------------------------------ 명화에서 길을 찾다 }
			\block 
			{■  명화에서 길을 찾다 }
			{				
			명화에서 길을 찾다 강소연 시공아트 2019
			[중앙] 654.22-23
			[수정] 652.22-12
			[구덕] 654.22 10
			[남구]654.22-강55
			}


%	------------------------------------------------------------------------------ 그림으로 보는 불교 이야기 }
			\block 
			{■  그림으로 보는 불교 이야기 }
			{				
			그림으로 보는 불교 이야기 정병삼 풀빛 2000
			[중앙] 220.4-10 서고
			[금정] 654.22-정44그
			}

%	------------------------------------------------------------------------------ (재미있는) 우리 사찰의 벽화이야기 }
			\block
			{■  (재미있는) 우리 사찰의 벽화이야기 }
			{				
			(재미있는) 우리 사찰의 벽화이야기 권영한 전원문화사 2011
			[금정] 654.22-권64우
			}

%	------------------------------------------------------------------------------ 사찰불화 명작강의 }
			\block
			{■  사찰불화 명작강의 }
			{				
			사찰불화 명작강의 강소연 불광출판사 2016 
			[수정] 654.22 10
			}

%	------------------------------------------------------------------------------ 왕실 권력 그리고 불화 }
			\block
			{■  왕실 권력 그리고 불화 }
			{				
			왕실 권력 그리고 불화  김정희 세창출판사 2019
			[수정] 654.22 13
			}


	%	====== ====== ====== ====== ====== 
		\column{0.2}


%	------------------------------------------------------------------------------ 아두이노
			\block[
%						titleoffsetx =0mm,	 	%
%						titleoffsety=0mm,	 	% 
%						bodyoffsetx=00mm,	%
						bodyoffsety=20mm	 	%
%						titlewidthscale=2, 
%						bodywidthscale=1,
						titleleft, 
%						titlecenter, 
%						titleright,
%						bodyverticalshift=0mm  % 제목과 본과의 간격
%						roundedcorners=50, 
						linewidth=3mm,
%						titleinnersep=10mm, 
%						bodyinnersep=0mm
					]
					{■  아두이노}

%	------------------------------------------------------------------------------ 아두이노
			\block 
			{■  손에 잡히는 아두이노}
			{				
			[중앙] 569-24
			}

%	------------------------------------------------------------------------------ 두근두근 아두이노 공작소
			\block 
			{■  두근두근 아두이노 공작소 }
			{				
			[수정] 569 17
			}

%	------------------------------------------------------------------------------ 공필화 입문
			\block {■ 공필화 입문 }
			{				
			리강 저 평사리,2018
			[남구] 653.12-리12공-1-7 
			[금정] 공필화 : 정연하고 섬세한 화법에 속하여 교밀하면서도 정세한회화 652.5-김22공
			[강서]  공필화 김다예 디다아트,2016 652.59-김22
			}


	%	====== ====== ====== ====== ====== 
		\column{0.2}

%	------------------------------------------------------------------------------  한권으로 읽는 아함경
			\block [titleleft,
					linewidth=1mm]{■ 한권으로  읽는 아함경 }
		{
%[중앙] 
%[수정] 
%[구덕]
%[남구]
[부전] 223.51-15
%[시민]
		}		


%	------------------------------------------------------------------------------  아함경
			\block [titleleft,
					linewidth=1mm]{■ 아함경 }
		{
[중앙] 220.8-2-4
%[수정] 
%[구덕]
%[남구] 
%[부전] 
%[시민]
		}		


%	------------------------------------------------------------------------------  한권으로 읽는 빠알리 성전
			\block [titleleft,
					linewidth=1mm]{■ 한권으로 읽는 빠알리 성전 }
		{
[중앙] 223.579-4
%[수정] 
%[구덕]
[남구] 223.1-일62빠
%[부전] 
%[시민]
		}		



%	------------------------------------------------------------------------------ 붓다의 가르침과 팔정도
			\block [titleleft,
					linewidth=1mm]{■ 붓다의 가르침과 팔정도 }
		{
[중앙] 221-21
%[수정] 
%[구덕]
%[남구] 
[부전] 221-50
%[시민]
		}		





	%	====== ====== ====== ====== ====== 
		\column{0.2}

%	------------------------------------------------------------------------------ 화엄경
			\block 
			{■ (한 권으로 읽는)화엄경 : 80권 40품 전체 내용 요약 풀이(만화) 임기준}
			{				
			[중앙] 223.55-19
			[수정]  233.55-5
		}




%	------------------------------------------------------------------------------ 이것이 간화선이다 : 무비 스님의 서장 강설
			\block 
			{■ 이것이 간화선이다 }
			{				
			[중앙]  228.7-147
			[수정] 228.7-49
			[남구] 224.81-대94
		}

%	------------------------------------------------------------------------------ 발심수행장
			\block 
			{■  발심수행장 : 공파스님 }
			{				
			[중앙] 225.74-4 
			[수정] 225.74-2 
			[남구] 225.74공892
			}


%	------------------------------------------------------------------------------ 발심수행장
			\block 
			{■  찬불가 }
			{				
			뭇소리 찬불가 찬불가악보집 박범훈 396쪽 2014년 4월 16일 민속원
			[시민] 통일법요집 227.1-2 
			[시민] 어린이 법요 찬불가집 227.1-4 
		}


	%	====== ====== ====== ====== ====== 
		\column{0.2}

%	------------------------------------------------------------------------------ 돌계집이 애를 낳는구나 : 제불조사의 선시, 깨달음의 노래
			\block 
			{■  돌계집이 애를 낳는구나 }
			{				
			돌계집이 애를 낳는구나 : 제불조사의 선시, 깨달음의 노래	이계묵 지음 비움과 소통 2014년
			[부전] 224.2-297  
			[구포] 224.2-132
		}



%	------------------------------------------------------------------------------ 행복과 평화를 주는 가르침
			\block 
			{■ 행복과 평화를 주는 가르침 }
		{
[중앙] 
[수정] 
[구덕]
[남구] 
[부전] 
[시민] 233-85
[해운대] 223.1-3

		}		





%	------------------------------------------------------------------------------ 반야참회
			\block 
			{■ 반야참회 내가원하는 것을 이루게 하는 힘 }
			{				
해룡 저 불광출판사
[초읍] 224.81-123
[부전] 224.81-92 서고
[중앙] 224.81-114 3층
		}


	%	====== ====== ====== ====== ====== 
		\column{0.2}





	\end{columns}




\end{document}


		\begin{huge}
		\end{huge}

		\begin{LARGE}
		\end{LARGE}

		\begin{Large}
		\end{Large}

		\begin{large}
		\end{large}

