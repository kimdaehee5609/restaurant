%	-------------------------------------------------------------------------------
%
%		작성		2020년 
%				10월 
%				2일 
%				금요일
%
%
%
%
%
%
%	-------------------------------------------------------------------------------

%\documentclass[10pt,xcolor=pdftex,dvipsnames,table]{beamer}
%\documentclass[10pt,blue,xcolor=pdftex,dvipsnames,table,handout]{beamer}
%\documentclass[14pt,blue,xcolor=pdftex,dvipsnames,table,handout]{beamer}
\documentclass[aspectratio=1610,17pt,xcolor=pdftex,dvipsnames,table,handout]{beamer}

			\geometry{paper=a5paper}

		% Font Size
		%	default font size : 11 pt
		%	8,9,10,11,12,14,17,20
		%
		% 	put frame titles 
		% 		1) 	slideatop
		%		2) 	slide centered
		%
		%	navigation bar
		% 		1)	compress
		%		2)	uncompressed
		%
		%	Color
		%		1) blue
		%		2) red
		%		3) brown
		%		4) black and white	
		%
		%	Output
		%		1)  	[default]	
		%		2)	[handout]		for PDF handouts
		%		3) 	[trans]		for PDF transparency
		%		4)	[notes=hide/show/only]

		%	Text and Math Font
		% 		1)	[sans]
		% 		2)	[sefif]
		%		3) 	[mathsans]
		%		4)	[mathserif]


		%	---------------------------------------------------------	
		%	슬라이드 크기 설정 ( 128mm X 96mm )
		%	---------------------------------------------------------	
%			\setbeamersize{text margin left=2mm}
%			\setbeamersize{text margin right=2mm}

	%	========================================================== 	Package
		\usepackage{kotex}						% 한글 사용
		\usepackage{amssymb,amsfonts,amsmath}	% 수학 수식 사용
		\usepackage{color}					%
		\usepackage{colortbl}					%


	%		========================================================= 	note 옵션인 
	%			\setbeameroption{show only notes}
		

	%		========================================================= 	Theme

		%	---------------------------------------------------------	
		%	전체 테마
		%	---------------------------------------------------------	
		%	테마 명명의 관례 : 도시 이름
%			\usetheme{default}			%
%			\usetheme{Madrid}    		%
%			\usetheme{CambridgeUS}    	% -red, no navigation bar
%			\usetheme{Antibes}			% -blueish, tree-like navigation bar

		%	----------------- table of contents in sidebar
			\usetheme{Berkeley}		% -blueish, table of contents in sidebar
									% 개인적으로 마음에 듬

%			\usetheme{Marburg}			% - sidebar on the right
%			\usetheme{Hannover}		% 왼쪽에 마크
%			\usetheme{Berlin}			% - navigation bar in the headline
%			\usetheme{Szeged}			% - navigation bar in the headline, horizontal lines
%			\usetheme{Malmoe}			% - section/subsection in the headline

%			\usetheme{Singapore}
%			\usetheme{Amsterdam}

		%	---------------------------------------------------------	
		%	색 테마
		%	---------------------------------------------------------	
%			\usecolortheme{albatross}	% 바탕 파란
%			\usecolortheme{crane}		% 바탕 흰색
%			\usecolortheme{beetle}		% 바탕 회색
%			\usecolortheme{dove}		% 전체적으로 흰색
%			\usecolortheme{fly}		% 전체적으로 회색
%			\usecolortheme{seagull}	% 휜색
%			\usecolortheme{wolverine}	& 제목이 노란색
%			\usecolortheme{beaver}

		%	---------------------------------------------------------	
		%	Inner Color Theme 			내부 색 테마 ( 블록의 색 )
		%	---------------------------------------------------------	

%			\usecolortheme{rose}		% 흰색
%			\usecolortheme{lily}		% 색 안 칠한다
%			\usecolortheme{orchid} 	% 진하게

		%	---------------------------------------------------------	
		%	Outter Color Theme 		외부 색 테마 ( 머리말, 고리말, 사이드바 )
		%	---------------------------------------------------------	

%			\usecolortheme{whale}		% 진하다
%			\usecolortheme{dolphin}	% 중간
%			\usecolortheme{seahorse}	% 연하다

		%	---------------------------------------------------------	
		%	Font Theme 				폰트 테마
		%	---------------------------------------------------------	
%			\usfonttheme{default}		
			\usefonttheme{serif}			
%			\usefonttheme{structurebold}			
%			\usefonttheme{structureitalicserif}			
%			\usefonttheme{structuresmallcapsserif}			



		%	---------------------------------------------------------	
		%	Inner Theme 				
		%	---------------------------------------------------------	

%			\useinnertheme{default}
			\useinnertheme{circles}		% 원문자			
%			\useinnertheme{rectangles}		% 사각문자			
%			\useinnertheme{rounded}			% 깨어짐
%			\useinnertheme{inmargin}			




		%	---------------------------------------------------------	
		%	이동 단추 삭제
		%	---------------------------------------------------------	
%			\setbeamertemplate{navigation symbols}{}

		%	---------------------------------------------------------	
		%	문서 정보 표시 꼬리말 적용
		%	---------------------------------------------------------	
%			\useoutertheme{infolines}


			
	%	---------------------------------------------------------- 	배경이미지 지정
%			\pgfdeclareimage[width=\paperwidth,height=\paperheight]{bgimage}{./fig/Chrysanthemum.jpg}
%			\setbeamertemplate{background canvas}{\pgfuseimage{bgimage}}

		%	---------------------------------------------------------	
		% 	본문 글꼴색 지정
		%	---------------------------------------------------------	
%			\setbeamercolor{normal text}{fg=purple}
%			\setbeamercolor{normal text}{fg=red!80}	% 숫자는 투명도 표시


		%	---------------------------------------------------------	
		%	itemize 모양 설정
		%	---------------------------------------------------------	
%			\setbeamertemplate{items}[ball]
%			\setbeamertemplate{items}[circle]
%			\setbeamertemplate{items}[rectangle]






		\setbeamercovered{dynamic}





		% --------------------------------- 	문서 기본 사항 설정
		\setcounter{secnumdepth}{3} 		% 문단 번호 깊이
		\setcounter{tocdepth}{3} 			% 문단 번호 깊이




% ------------------------------------------------------------------------------
% Begin document (Content goes below)
% ------------------------------------------------------------------------------
	\begin{document}
	

			\title{ 김대희 식당 }
			\author{ 김대희 }
			\date{ 2020년 
					12월
					23일 
					수요일 }


% -----------------------------------------------------------------------------
%		개정 내용
% -----------------------------------------------------------------------------
%
%		2020년 6월 28일 첫제작
%		2020.10.23 부산 국수집 추가
%		2020.10.26 면과장 part 추가
%


	%	==========================================================
	%
	%	---------------------------------------------------------- 	page	1
		\begin{frame}[plain]
		\titlepage
		\end{frame}

	%	---------------------------------------------------------- 	page	2
		\begin{frame} [plain]{목차}
		\tableofcontents%


%			\setlength{\leftmargini}{ 2em}			
%			\begin{itemize}
%
%				\item [part1] \ref{part1}	부산 국수집
%%\label{part1} 	%  부산국수집
%				\item [part2] \ref{part2}	면과장
%%\label{part2} 	%  면과장
%				\item [part3] \ref{part3}	부산
%%\label{part3} 	%  부산
%				\item [part4] \ref{part4}	울산
%%\label{part4} 	%  울산
%				\item [part5] \ref{part5}	가보고 싶은 곳
%%\label{part5} 	%  가보고 싶은 곳
%%				\item [part6] \ref{part6}	일반사이트		
%%				\item [part7] \ref{part7}	쇼핑 사이트		
%
%			\end{itemize}
%


		\end{frame}

	%	---------------------------------------------------------- 	page	3
		\begin{frame} [plain]
		\end{frame}
	%	---------------------------------------------------------- 	page	4
		\begin{frame} [plain]
		\end{frame}


	%	========================================================== 부산 국수집
		\part{ 부산 국수집}
		\frame{\partpage}


		\begin{frame} [plain]{목차}
		\tableofcontents%

\label{part1} 	%  부산 국수집

		\end{frame}
		

	%	---------------------------------------------------------- 용호동 양푼이 국수 }
	%		Frame
	%	----------------------------------------------------------
		\section{ 용호동 양푼이 국수 }
		\begin{frame} [t,plain]
		\frametitle{ 용호동 양푼이 국수 }
			\begin{block} { 용호동 양푼이 국수 }
			\setlength{\leftmargini}{4em}			
			\begin{itemize}
				\item [지역] 용호동	
				\item [명칭] 양푼이국수	
				\item [주소] 남구 용호로34 일신프리빌리지
			\end{itemize}
			\end{block}						
		\end{frame}						

	%	---------------------------------------------------------- 서면 화전국수	}
	%		Frame
	%	----------------------------------------------------------
		\section{ 서면 화전국수	}
		\begin{frame} [t,plain]
		\frametitle{ 서면 화전국수	}
			\begin{block} { 서면 화전국수	}
			\setlength{\leftmargini}{4em}			
			\begin{itemize}
				\item [지역] 서면	
				\item [명칭] 서면 화전국수	
				\item [주소] 진구 중앙대로 702번길 17-2
			\end{itemize}
			\end{block}						
		\end{frame}						


	%	---------------------------------------------------------- 수영 둔내막국수	}
	%		Frame
	%	----------------------------------------------------------
		\section{ 수영 둔내막국수	}
		\begin{frame} [t,plain]
		\frametitle{ 수영 둔내막국수	}
			\begin{block} { 수영 둔내막국수	}
			\setlength{\leftmargini}{4em}			
			\begin{itemize}
				\item [지역] 수영	
				\item [명칭] 수영 둔내막국수	
				\item [주소] 수영구 수영로 725번길62-3
			\end{itemize}
			\end{block}						
		\end{frame}						

	%	---------------------------------------------------------- 소문난 주문진 막국수	}
	%		Frame
	%	----------------------------------------------------------
		\section{ 소문난 주문진 막국수	}
		\begin{frame} [t,plain]
		\frametitle{ }
			\begin{block} {소문난 주문진 막국수	}
			\setlength{\leftmargini}{4em}			
			\begin{itemize}
				\item [지역] 사직동	
				\item [명칭] 소문난 주문진 막국수	
				\item [주소] 동래구 사직로 58번길8
			\end{itemize}
			\end{block}						
		\end{frame}						


	%	---------------------------------------------------------- 천서리아가네 막국수	}
	%		Frame
	%	----------------------------------------------------------
		\section{ 천서리아가네 막국수	}
		\begin{frame} [t,plain]
		\frametitle{ }
			\begin{block} {천서리아가네 막국수	}
			\setlength{\leftmargini}{4em}			
			\begin{itemize}
				\item [지역] 온천장	
				\item [명칭] 천서리아가네 막국수	
				\item [주소] 금정구 식물원로27
			\end{itemize}
			\end{block}						
		\end{frame}						


	%	----------------------------------------------------------  기장 원조열무국수	}
	%		Frame
	%	----------------------------------------------------------
		\section{ 기장 원조열무국수	}
		\begin{frame} [t,plain]
		\frametitle{ }
			\begin{block} {기장 원조열무국수	}
			\setlength{\leftmargini}{4em}			
			\begin{itemize}
				\item [지역] 기장	
				\item [명칭] 원조열무국수	
				\item [주소] 기장읍 당사로5길11-3
			\end{itemize}
			\end{block}						
		\end{frame}						


	%	----------------------------------------------------------  대연동 경호강어탕국수	}
	%		Frame
	%	----------------------------------------------------------
		\section{ 대연동 경호강어탕국수	}
		\begin{frame} [t,plain]
		\frametitle{ }
			\begin{block} {대연동 경호강어탕국수	}
			\setlength{\leftmargini}{4em}			
			\begin{itemize}
				\item [지역] 대연동	
				\item [명칭] 경호강어탕국수	
				\item [주소] 남구 못골로78
			\end{itemize}
			\end{block}						
		\end{frame}						


	%	----------------------------------------------------------  대저 할매국수	}
	%		Frame
	%	----------------------------------------------------------
		\section{ 대저 할매국수	}
		\begin{frame} [t,plain]
		\frametitle{ }
			\begin{block} {대저 할매국수	}
			\setlength{\leftmargini}{4em}			
			\begin{itemize}
				\item [지역] 대저	
				\item [명칭] 대저 할매국수	
				\item [주소] 강서구 대저중앙로337
			\end{itemize}
			\end{block}						
		\end{frame}						


	%	----------------------------------------------------------  초읍 원가네	}
	%		Frame
	%	----------------------------------------------------------
		\section{ 초읍 원가네	}
		\begin{frame} [t,plain]
		\frametitle{ }
			\begin{block} {초읍 원가네	}
			\setlength{\leftmargini}{4em}			
			\begin{itemize}
				\item [지역] 초읍	
				\item [명칭] 초읍 원가네	
				\item [주소] 부산진구 성지로145
			\end{itemize}
			\end{block}						
		\end{frame}						


	%	----------------------------------------------------------  연산동 다미국수	}
	%		Frame
	%	----------------------------------------------------------
		\section{ 연산동 다미국수	}
		\begin{frame} [t,plain]
		\frametitle{ }
			\begin{block} {연산동 다미국수	}
			\setlength{\leftmargini}{4em}			
			\begin{itemize}
				\item [지역] 연산동	
				\item [명칭] 연산동 다미국수	
				\item [주소] 연제구 과정로200-1
			\end{itemize}
			\end{block}						
		\end{frame}						

	%	----------------------------------------------------------  이원화 구포국수	}
	%		Frame
	%	----------------------------------------------------------
		\section{ 이원화 구포국수	}
		\begin{frame} [t,plain]
		\frametitle{ }
			\begin{block} {	이원화 구포국수	}
			\setlength{\leftmargini}{4em}			
			\begin{itemize}
				\item [지역] 구포	
				\item [명칭] 이원화 구포국수	
				\item [주소] 북구 구포시장1길 6
			\end{itemize}
			\end{block}						
		\end{frame}						


	%	----------------------------------------------------------  수영 회국수	}
	%		Frame
	%	----------------------------------------------------------
		\section{ 수영 회국수	}
		\begin{frame} [t,plain]
		\frametitle{ }
			\begin{block} {	수영 회국수	}
			\setlength{\leftmargini}{4em}			
			\begin{itemize}
				\item [지역] 수영	
				\item [명칭] 수영 회국수	
				\item [주소] 수영구 수영로660번길56	
			\end{itemize}
			\end{block}						
		\end{frame}						




	%	----------------------------------------------------------  수영 나룻터국수	}
	%		Frame
	%	----------------------------------------------------------
		\section{ 수영 나룻터국수	}
		\begin{frame} [t,plain]
		\frametitle{ }
			\begin{block} {	수영 나룻터국수	}
			\setlength{\leftmargini}{4em}			
			\begin{itemize}
				\item [지역] 수영	
				\item [명칭] 수영 나룻터국수	
				\item [주소] 수영구 수영로741번길25
			\end{itemize}
			\end{block}						
		\end{frame}						


	%	----------------------------------------------------------  김해 대동 할매 국수	}
	%		Frame
	%	----------------------------------------------------------
		\section{ 김해 대동 할매 국수	}
		\begin{frame} [t,plain]
		\frametitle{ }
			\begin{block} {	김해 대동 할매 국수	}
			\setlength{\leftmargini}{4em}			
			\begin{itemize}
				\item [지역] 김해	
				\item [명칭] 김해 대동 할매 국수	
				\item [주소] 대동면 동남로45번길8
			\end{itemize}
			\end{block}						
		\end{frame}						


	%	----------------------------------------------------------  해운대 면옥향천	}
	%		Frame
	%	----------------------------------------------------------
		\section{ 해운대 면옥향천	}
		\begin{frame} [t,plain]
		\frametitle{ }
			\begin{block} {	해운대 면옥향천	}
			\setlength{\leftmargini}{4em}			
			\begin{itemize}
				\item [지역] 해운대	
				\item [명칭] 해운대 면옥향천	
				\item [주소] 해운대구 해운대로383번길26
			\end{itemize}
			\end{block}						
		\end{frame}						

	%	----------------------------------------------------------  남산동 구포촌국수	}
	%		Frame
	%	----------------------------------------------------------
		\section{ 남산동 구포촌국수	}
		\begin{frame} [t,plain]
		\frametitle{ }
			\begin{block} {	남산동 구포촌국수	}
			\setlength{\leftmargini}{4em}			
			\begin{itemize}
				\item [지역] 남산동	
				\item [명칭] 남산동 구포촌국수	
				\item [주소] 금정구 금생로490
			\end{itemize}
			\end{block}						
		\end{frame}						


	%	----------------------------------------------------------  중앙동 중앙모밀	}
	%		Frame
	%	----------------------------------------------------------
		\section{ 중앙동 중앙모밀	}
		\begin{frame} [t,plain]
		\frametitle{ }
			\begin{block} {	중앙동 중앙모밀	}
			\setlength{\leftmargini}{4em}			
			\begin{itemize}
				\item [지역] 중앙동	
				\item [명칭] 중앙모밀	
				\item [주소] 중구 중앙대로49번길 9-1
			\end{itemize}
			\end{block}						
		\end{frame}						




	%	========================================================== 면과장
	%
	%	---------------------------------------------------------- 	page 	1
		\part{ 면과장 }
		\frame{\partpage}

	%	---------------------------------------------------------- 	page 	2
		\begin{frame} [plain]{목차}
		\tableofcontents%

\label{part2} 	%  면과장

		\end{frame}
		
	%	---------------------------------------------------------- 	page 	3
		\begin{frame} [plain]
		\end{frame}
	%	---------------------------------------------------------- 	page 	4
		\begin{frame} [plain]
		\end{frame}



	% ---------------------------------------------------------- 					2020.1.30 부산덕포시장 또와분식				}
	% Frame									
	% ---------------------------------------------------------- page 1									
		\section{2020.1.30 부산덕포시장 또와분식 }								
		\begin{frame} [t,plain]								
		\frametitle{2020.1.30 부산덕포시장 또와분식}								
			\begin{block} {2020.1.30 부산덕포시장 또와분식}							
			\setlength{\leftmargini}{4em}							
			\begin{itemize}							
				\item [지역] 		덕포동				
				\item [명칭] 		또와분식				
				\item [주소]		부산 사상구 사상로289-6				
				\item [가격]		보리밥 5,000 손칼국수 4,500				
				\item [휴무]						
				\item [평가]						
			\end{itemize}							
			\end{block}							
		\end{frame}								
										



	%	---------------------------------------------------------- 2020.04.03 수영 회국수
	%		Frame
	%	---------------------------------------------------------- 	page 	2
		\section{ 2020.04.03 수영 회국수 }
		\begin{frame} [t,plain]
		\frametitle{ 2020.04.03 수영 회국수 }
			\begin{block} { 2020.04.03 수영 회국수 }
			\setlength{\leftmargini}{4em}			
			\begin{itemize}
				\item [지역] 수영 
				\item [명칭] 수영 회국수
				\item [주소] 수영구 수영로680번길56
				\item [전화] 051) 752 - 5788
				\item [시간] 11시 30 - 21시 00 (9시)
				\item [휴무] 
				\item [평가] 
			\end{itemize}
			\end{block}						
		\end{frame}						



	%	---------------------------------------------------------- 2020.04.22 	명장동 태양반점
	%		Frame
	%	---------------------------------------------------------- 	page 	3
		\section 	{2020.04.22 	명장동 태양반점}
		\begin{frame} [t,plain]
		\frametitle 			{2020.04.22 	명장동 태양반점}
			\begin{block} 	{2020.04.22 	명장동 태양반점}
			\setlength{\leftmargini}{4em}			
			\begin{itemize}
				\item [지역] 명장동
				\item [명칭] 태양반점
				\item [주소] 부산 동래구 명안로71번길 11-3
				\item [결제] 카드결제 가능
				\item [주차장] 따로 없습니다.  가게앞 주차가능
				\item [평가] 
			\end{itemize}
			\end{block}						
		\end{frame}						

	%	---------------------------------------------------------- 2020.07.01 초량 금수각
	%		Frame
	%	---------------------------------------------------------- 	page 	4
		\section{ 2020.07.01 초량 금수각 }
		\begin{frame} [t,plain]
		\frametitle{ 2020.07.01 초량 금수각 }
			\begin{block} { 2020.07.01 초량 금수각 }
			\setlength{\leftmargini}{4em}			
			\begin{itemize}
				\item [지역] 초량동
				\item [명칭] 금수각
				\item [주소] 초량로119
				\item [전화] 051) 442 - 2340
				\item [시간] 10시 30 - 20시 00 (8시)
				\item [휴무] 매주 일요일
				\item [평가] 
			\end{itemize}
			\end{block}						
		\end{frame}						

	%	---------------------------------------------------------- 2020.09.05 곱창 국수 }
	%		Frame
	%	---------------------------------------------------------- 	page 	1
		\section{ 2020.09.05 곱창 국수 }
		\begin{frame} [t,plain]
		\frametitle{ 2020.09.05 곱창 국수 }
			\begin{block} { 2020.09.05 곱창 국수 }
			\setlength{\leftmargini}{4em}			
			\begin{itemize}
				\item [지역] 서면
				\item [명칭] 카이시소소
				\item [주소] 
				\item [평가] 
			\end{itemize}
			\end{block}						
		\end{frame}						


	%	---------------------------------------------------------- 2020.11.04 당감동 복돌네통닭
	%		Frame
	%	---------------------------------------------------------- 	page 	2
		\section{ 2020.11.04 당감동 복돌네통닭}
		\begin{frame} [t,plain]
		\frametitle{ 2020.11.04 당감동 복돌네통닭}
			\begin{block} { 2020.11.04 당감동 복돌네통닭}
			\setlength{\leftmargini}{4em}			
			\begin{itemize}
				\item [지역] 당감동
				\item [명칭] 복돌네통닭
				\item [주소] 당감로16번길 42
				\item [전화] 051)897-6579
				\item [평가] 
			\end{itemize}
			\end{block}						
		\end{frame}						

	%	---------------------------------------------------------- 	page 	3
		\begin{frame} [t,plain]
		\end{frame}						
	%	---------------------------------------------------------- 	page 	4
		\begin{frame} [t,plain]
		\end{frame}						


	%	========================================================== 부산
		\part{ 부산 }
		\frame{\partpage}

\label{part3} 	%  부산

		\begin{frame} [plain]{목차}
		\tableofcontents%
		\end{frame}



	%	========================================================== 울산
		\part{ 울산 }
		\frame{\partpage}
		
\label{part4} 	%  울산

		\begin{frame} [plain]{목차}
		\tableofcontents%
		\end{frame}

	%	---------------------------------------------------------- 울산 천안문
	%		Frame
	%	----------------------------------------------------------
		\section{ 울산 천안문 }
		\begin{frame} [t,plain]
		\frametitle{ }
			\begin{block} { 울산 천안문 }
			\setlength{\leftmargini}{4em}			
			\begin{itemize}
				\item [분류] 중식
				\item [상호] 울산 천안문
				\item [전번] 052) 232 2345
				\item [주소] 울산 동구 화진2길 39
				\item [주차장]  인근 길가
				\item [결제] 카드 가능
				\item [주메뉴] 짜장 2000
			\end{itemize}
			\end{block}						

		\end{frame}						




	%	========================================================== 가보고 싶은  곳 }
		\part{ 가보고 싶은  곳 }
		\frame{\partpage}
		
\label{part5} 	%  가보고 싶은  곳 }

		\begin{frame} [plain]{목차}
		\tableofcontents%
		\end{frame}




% ------------------------------------------------------------------------------
% End document
% ------------------------------------------------------------------------------





\end{document}


